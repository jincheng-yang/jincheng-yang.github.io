%% start of file `template.tex'.
%% Copyright 2006-2015 Xavier Danaux (xdanaux@gmail.com).
%
% This work may be distributed and/or modified under the
% conditions of the LaTeX Project Public License version 1.3c,
% available at http://www.latex-project.org/lppl/.



\documentclass[11pt,letterpaper,roman]{moderncv} % Font sizes: 10, 11, or 12; paper sizes: a4paper, letterpaper, a5paper, legalpaper, executivepaper or landscape; font families: sans or roman

\moderncvstyle{casual} % CV theme - options include: 'casual' (default), 'classic', 'oldstyle' and 'banking'
\moderncvcolor{utnavy} % CV color - options include: 'blue' (default), 'orange', 'green', 'red', 'purple', 'grey' and 'black'

\usepackage{lipsum} % Used for inserting dummy 'Lorem ipsum' text into the template

\usepackage[scale=0.8]{geometry} % Reduce document margins
\setlength{\hintscolumnwidth}{5.8em} % Uncomment to change the width of the dates column
%\setlength{\makecvtitlenamewidth}{10cm} % For the 'classic' style, uncomment to adjust the width of the space allocated to your name

\renewcommand*{\namefont}{\fontsize{20}{20}\mdseries\upshape}
\renewcommand*{\footsymbol}{\qquad}
% \newcommand{\mysubsection}[1]{\subsection{\colorbox{color2!60!white}{\color{white}#1}}}
\newcommand{\mysubsection}[1]{\subsection{\underline{#1}}}

\usepackage{amsmath,amssymb, amsthm, enumerate, tikz-cd, etoolbox, colortbl, dsfont}
\usepackage{spectralsequences, float}
\usepackage{graphicx}
\usetikzlibrary{calc, intersections}
\usepackage[justification=centering]{caption}
\usepackage{xfrac}
\usepackage{xstring}
\usepackage[calc,showdow,english]{datetime2}
\AfterPreamble{
  \usepackage{doi}
}

\makeatletter
\renewcommand*{\bibliographyitemlabel}{\@biblabel{\arabic{enumiv}}}
\renewcommand*{\bibliographyhead}[1]{\section{Publications and Preprints}}
\makeatother

\DTMnewdatestyle{MMyyyy}{%
\renewcommand{\DTMdisplaydate}[4]{%
  \DTMshortmonthname{##2}\space%  (full) Month
  \number##1%                       year
}%
\renewcommand{\DTMDisplaydate}{\DTMdisplaydate}%
}
\DTMsetdatestyle{MMyyyy}

\setlength{\footskip}{55pt}

\newcounter{section} \setcounter{section}{1} \newtheorem{theorem}{Theorem}

 %\newtheorem{lemma}[theorem]{Lemma}

\newtheorem*{theorem*}{Theorem}

% \theoremstyle{definition}
% \newtheorem{definition}[theorem]{Definition}
% \newtheorem{xca}[theorem]{Exercise}
\newtheorem{proposition}[theorem]{Proposition}
% \newtheorem{corollary}[theorem]{Corollary}

% \theoremstyle{remark}
% \newtheorem{remark}[theorem]{Remark}

% \newcolumntype{a}{>{\columncolor{gray!25}}c}

% \numberwithin{equation}{section}


\usepackage{etoolbox}% http://ctan.org/pkg/etoolbox
\makeatletter
\patchcmd{\makeletterhead}% <cmd>
  {\raggedright \@opening}% <search>
  {\@opening}% <replace>
  {}{}% <success><failure>
\makeatother


%----------------------------------------------------------------------------------------
%	NAME AND CONTACT INFORMATION SECTION
%----------------------------------------------------------------------------------------

\firstname{Jincheng} % Your first name
\familyname{Yang} % Your last name

% All information in this block is optional, comment out any lines you don't need
\title{Curriculum Vitae}
\address{Johns Hopkins University $\bullet$ 3400 N Charles St $\bullet$ Baltimore, MD 21218}
\homepage{jincheng-yang.github.io}
\email{jincheng@jhu.edu}
% The first argument is the url for the clickable link, the second argument is the url displayed in the template - this allows special characters to be displayed such as the tilde in this example
%\extrainfo{additional information}
%\photo[70pt][0.4pt]{pictures/picture} % The first bracket is the picture height, the second is the thickness of the frame around the picture (0pt for no frame)
%\quote{"A witty and playful quotation" - John Smith}

%----------------------------------------------------------------------------------------

\begin{document}
%\begin{CJK*}{UTF8}{gbsn}                          % to typeset your resume in Chinese using CJK
%-----       resume       ---------------------------------------------------------
%\makecvtitle

%-----       letter       ---------------------------------------------------------
% recipient data
\recipient{School}{Address}
\date{\today}
\opening{Dear Search Committee,}
\closing{Sincerely,}
% \makelettertitle
% I am writing to apply to the position of Assistant Adjunct Professorship in the Mathematics Department at the University of California, Los Angeles.  I am currently a graduate student at the University of Texas at Austin under the supervision of Andrew Blumberg.  I expect to complete my Ph.D. in May 2021.

My research is in computational equivariant homotopy theory, applied to the modular representation theory of finite groups.  In my thesis work, I used homotopy-theoretic methods to compute the group of endotrivial modules for certain $p$-groups, thereby providing new insights into classical computations and results in modular representation theory.  In future work, I plan to extend my computations to other classes of groups, and I also plan to study other problems in representation theory by constructing and computing novel homotopy invariants. 

I believe that the environment at UCLA would allow my research program to flourish. In particular, my research interests are most closely aligned with those of Professors Mike Hill, Paul Balmer, and Rapha\"el Rouquier.  Furthermore, there is also a strong postdoctoral group, including Christy Hazel and Hood Chatham, as well as the graduate student groups in algebra and topology that I could possibly collaborate with.

Beyond my research program, I would be an active participant and contributor to the department. I have also given various seminar and invited talks, and I have also organized several learning seminars throughout my time at UT Austin.

I am also committed to actively promoting and supporting diversity, inclusivity, and equity in mathematics through both education and outreach.  I would also use my previous experiences to establish new educational or diversity-focused initiatives to support the department.

I have arranged for letters of recommendation to be sent via MathJobs.org from Andrew Blumberg and 
Dan Isaksen, as well as a letter from Mark Daniels,
attesting to my teaching abilities. I am also submitting the AMS cover
sheet, my CV, research statement, teaching statement, and diversity statement. 

I appreciate your time and consideration and look forward to hearing from you.
% \makeletterclosing

% \newpage

%----------------------------------------------------------------------------------------
%	CURRICULUM VITAE
%----------------------------------------------------------------------------------------

\makecvtitle % Print the CV title

% \section{About Me}

\section{Appointments}

\newcommand{\appointment}[6]{\cventry{#5}{#1}{#2}{#6}{}{}}
\appointment{Member}{Institute for Advanced Study}{2023-09-01}{2024-07-31}{2024--2025}{Princeton, NJ}
\appointment{L.E. Dickson Instructor}{University of Chicago}{2022-09-01}{2024-08-31}{2022--2024}{Chicago, IL}
\appointment{Visiting Scholar}{Georgia Institute of Technology}{2017-02-15}{2017-05-15}{2017}{Atlanta, GA}

\section{Education}

\newcommand{\education}[8]{\cventry{#5}{#1}{#2}{#6}{}{#8}}
\education{Ph.D. in Mathematics}{The University of Texas at Austin}{2017-08-15}{2022-05-30}{2017-2022}{Austin, TX}{doctor}{}%Advisor: Prof. Luis A. Caffarelli and Prof. Alexis F. Vasseur}
\cvitem{}{Advisor: Luis Caffarelli and Alexis Vasseur}
% \education{Visiting Honors Student Program}{Georgia Institute of Technology}{2016-01-09}{2016-05-12}{2016}{Atlanta, GA}{non-degree}{}
% \education{Visiting Undergraduate Student}{Columbia University}{2015-01-20}{2015-05-20}{2015}{New York City, NY}{non-degree}{}
\education{B.Sc. in Mathematics and Applied Mathematics \textnormal{(Honors Program)}}{Xi'an Jiaotong University}{2013-08-18}{2017-06-24}{2013-2017}{China}{bachelor}{}
\education{Special Class for the Gifted Young}{Xi'an Jiaotong University}{2011-08-30}{2013-07-10}{2011-2013}{China}{non-degree}{}


\section{Research Interests}
\cventry{\sffamily APDE}{Analysis and Partial Differential Equations}{dynamical systems, fluid dynamics, kinetic theory}{}{}{Euler equation, Navier--Stokes equation, Boltzmann equation, Fokker--Planck equation.}
\cventry{\sffamily MSOR}{Management Science and Operation Research}{distributionally robust optimization, optimal mass transportation, risk measure}{}{}{Inventory problem, risk management, portfolio optimization.}

% \cvitem{}{Partial differential equations, fluid dynamics, Euler equation, Navier--Stokes equation.}

% \section{Publications and Preprints}
\nocite{coiculescu2025,cao2024,yang2022a,martinez2025,yang2025,zhang2024b,vasseur2024,zhang2024,vasseur2023,yang2022a,yang2022,vasseur2021,lin2020,yang2018}
\bibliographystyle{unsrturl}
\bibliography{../../_data/publication.bib}


\cvitem{Title}{\emph{Partial regularity results for the three-dimensional incompressible Navier--Stokes equation}}
\cvitem{Advisors}{Prof. Luis Caffarelli and Prof. Alexis Vasseur}
\cvitem{Description}{We show a series of works of some regularity results on the incompressible Navier–Stokes equation in dimension three.}

\subsection{Undergraduate Thesis}

\cvitem{Title}{\emph{Linear Inviscid Damping of a Shear Flow in a Half-Space and a Finite Channel}}
\cvitem{Advisors}{Prof. Dongsheng Li and Prof. Zhiwu Lin}
\cvitem{Description}{We show the decay rate for velocity and density variation to linearized Euler equations near stratified Couette flow under optimal regularity.}

\section{Honors and Awards}
\newcommand{\honor}[4]{\cvitem{#3 #4}{#1, \textit{#2}}}
\honor{Frank Gerth III Outstanding Dissertation Award}{UT Austin}{May}{2022}
\honor{University Graduate Continuing Fellowship}{UT Austin}{Mar}{2021}
\honor{Frank Gerth III Teaching Excellence Award}{UT Austin}{June}{2020}
\honor{Senate of College Council's TA of the Year}{UT Austin}{Apr}{2019}
\honor{Frank Gerth III Graduate Excellence Award}{UT Austin}{June}{2018}
% \honor{Outstanding Student (The highest honor in undergraduate, \sfrac{10}{30000})}{XJTU}{Oct}{2016}
% \honor{National Scholarship}{Ministry of Education, China}{Sept}{2016}


% \subsection{Conferences and Workshops}

% \newcommand{\conference}[6]{\cvitem{\DTMdate{#2}}{#4, \textit{#1}}}
% \conference{AMS Fall Southeastern Sectional Meeting}{2025-10-03}{2025-10-05}{Tulane University}{https://meetings.ams.org/math/fall2025se/meetingapp.cgi/}{conference}
\conference{AMS Fall Western Sectional Meeting}{2025-08-23}{2025-08-24}{University of Denver}{https://meetings.ams.org/math/fall2025w/meetingapp.cgi/}{conference}
\conference{ICCOPT 2025}{2025-07-19}{2025-07-24}{University of Southern California}{https://www.iccopt2025.org/home/}{conference}
\conference{2025 AWM Research Symposium}{2025-05-16}{2025-05-18}{University of Wisconsin, Madison}{https://awm-math.org/meetings/awm-research-symposium/}{conference}
\conference{SIAM Conference on Applications of Dynamical Systems (DS25)}{2025-05-11}{2025-05-15}{Denver, Colorado}{https://www.siam.org/conferences-events/past-event-archive/ds25/}{conference}
\conference{Integro-differential equations in many-particle interacting systems}{2025-04-14}{2025-04-18}{American Institute of Mathematics}{https://aimath.org/workshops/upcoming/manyparticle/}{conference}
\conference{Frontiers in Computational Mathematics}{2025-04-10}{2025-04-12}{The University of Texas at Austin}{https://engquistconference.oden.utexas.edu}{conference}
\conference{ShapiroFest: A Celebration of Alexander Shapiro’s Legacy in Stochastic Optimization}{2025-03-17}{2025-03-18}{Georgia Institute of Technology}{https://sites.gatech.edu/shapirofest/}{conference}
\conference{NSF-FRG Conference on Fluids and Computer Assisted Proofs}{2025-03-14}{2025-03-16}{Princeton University}{https://web.math.princeton.edu/~aionescu/FRGconference.html}{conference}
\conference{Informs Annual Meeting 2024}{2024-10-20}{2024-10-23}{Seattle}{https://meetings.informs.org/wordpress/seattle2024/}{conference}
\conference{EquaDiff 2024}{2024-06-10}{2024-06-14}{Karlstad University}{https://www.kau.se/en/equadiff}{conference}
\conference{Recent Advances in Nonlinear Partial Differential Equations}{2024-05-13}{2024-05-17}{University of Minnesota}{https://cse.umn.edu/math/events/recent-advances-nonlinear-partial-differential-equations}{conference}
\conference{AMS Spring Southeastern Sectional Meeting}{2024-03-23}{2024-03-24}{Florida State University}{https://meetings.ams.org/math/spring2024se/meetingapp.cgi/}{conference}
\conference{Small scale dynamics in incompressible fluid flows}{2023-11-06}{2023-11-10}{American Institute of Mathematics}{https://aimath.org/workshops/upcoming/smallscalefluid/}{conference}
\conference{8<sup>th</sup> Annual Meeting of the SIAM Central States Section}{2023-10-06}{2023-10-06}{University of Nebraska-Lincoln}{https://math.unl.edu/siam-2023}{conference}
\conference{10<sup>th</sup> International Congress on Industrial and Applied Mathematics}{2023-08-20}{2023-08-25}{Waseda University}{https://iciam2023.org/}{conference}
\conference{Shocking Developments: New Directions in Compressible and Incompressible Flows: A Conference in Honor of Alexis Vasseur’s 50<sup>th</sup> Birthday}{2023-06-26}{2023-06-30}{Max Planck Institute for Mathematics in the Sciences}{https://www.mis.mpg.de/calendar/conferences/2023/shocks.html}{conference}
\conference{International Conference on Partial Differential Equations and Applications in honor of the 70<sup>th</sup> birthday of Pierangelo Marcati}{2023-06-19}{2023-06-23}{Gran Sasso Science Institute}{https://indico.gssi.it/event/486/}{conference}
\conference{Recent Advances in Mathematical Fluid Dynamics}{2023-05-20}{2023-05-24}{Duke University}{https://sites.duke.edu/fluids/}{conference}
\conference{87<sup>th</sup> Midwest PDE Seminar}{2023-05-06}{2023-05-07}{University of Notre Dame}{https://sites.nd.edu/midwestpde23/}{conference}
\conference{Informs Annual Meeting 2022}{2022-10-16}{2022-10-19}{Indianapolis}{https://meetings.informs.org/wordpress/indianapolis2022/#_gl=1*vz8t1r*_gcl_au*MTgyMDY3MTAyOS4xNjk3NjM5NjUy}{conference}
\conference{Criticality and Stochasticity in Quasilinear Fluid Systems}{2022-05-02}{2022-05-06}{American Institute of Mathematics}{https://aimath.org/workshops/upcoming/stochasticfluid/}{conference}
\conference{Rigorous analysis of incompressible fluid models and turbulence}{2022-02-14}{2022-02-18}{Isaac Newton Institute for Mathematical Science}{https://www.newton.ac.uk/event/turw02/}{conference}
\conference{New Mechanisms for Regularity, Singularity, and Long Time Dynamics in Fluid Equations}{2021-07-26}{2021-07-30}{Banff International Research Station}{https://www.birs.ca/events/2021/5-day-workshops/21w5110/}{conference}
\conference{Summer Program in Partial Differential Equations 2021}{2021-05-17}{2021-05-28}{The University of Texas at Austin}{https://analysispde.ma.utexas.edu/summer-program-in-partial-differential-equations-2021/}{summer school}
\conference{AMS Spring Western Sectional Meeting}{2021-05-01}{2021-05-02}{San Francisco State University}{https://meetings.ams.org/math/spring2021w/meetingapp.cgi/}{conference}
\conference{AMS Spring Eastern Sectional Meeting}{2021-03-20}{2021-03-21}{Brown University}{https://meetings.ams.org/math/spring2021e/meetingapp.cgi/}{conference}
\conference{Virtual Summer Graduate School: Introduction to Water Waves}{2020-07-27}{2020-08-07}{Mathematical Sciences Research Institute}{https://www.msri.org/summer_schools/910}{summer school}
\conference{Summer School in Semiclassical Analysis}{2019-07-29}{2019-08-16}{Northwestern University}{https://sites.northwestern.edu/snap2019/}{summer school}
\conference{Summer School on Fluid Mechanics at the ICMAT}{2019-06-24}{2019-06-28}{Instituto de Ciencias Matemáticas}{https://www.icmat.es/node/1788}{summer school}
\conference{Workshop on Free Boundary Problems}{2019-05-23}{2019-05-24}{Columbia University}{https://people.math.umass.edu/~nguillen/Workshop2019/index.html}{conference}
\conference{On Nonlinear PDEs and their Applications}{2019-03-01}{2019-03-02}{The University of Texas at Austin}{https://web.ma.utexas.edu/NonlinearPDEs2019/austin/}{conference}


\section{Invited Talks}
\newcommand{\talk}[6]{\cventry{\DTMdate{#2}}{\textnormal{#4,} \textnormal{\textit{#3}}}{}{}{}{#5}}%{\DTMdate{#2}}{#4}{#5}}
\talk{Conditional Liouville theorem for the stationary Navier-Stokes equations}{2025-10-05}{Tulane University}{AMS Fall Southeastern Sectional Meeting}{}{We present a novel approach to the Liouville problem for the stationary Navier-Stokes equations. Under assumptions on the growth rate of the antiderivative of the velocity, we prove there exists no nontrivial solution to the three-dimensional stationary Navier-Stokes equation.}
\talk{The flexibility and rigidity of incompressible fluid equations}{2025-09-04}{Johns Hopkins University}{Applied Mathematics and Statistics Seminar}{}{In this talk, we will showcase a few recent developments in the mathematical analysis of incompressible fluid equations, highlighting both their flexibility and rigidity. For evolutionary Euler and Navier-Stokes equation, we discuss when an initial condition corresponds to a unique solution or weak solution. For stationary Euler and Navier-Stokes equations, we explore the conditions under which nontrivial solutions exist or do not exist.}
\talk{Conditional Liouville theorem for the stationary Navier-Stokes equations}{2025-08-24}{University of Denver}{AMS Fall Western Sectional Meeting}{}{We present a novel approach to the Liouville problem for the stationary Navier-Stokes equations. Under assumptions on the growth rate of the antiderivative of the velocity, we prove there exists no nontrivial solution to the three-dimensional stationary Navier-Stokes equation.}
\talk{Beyond absolute continuity: a new class of dynamic risk measures}{2025-07-24}{University of Southern California}{ICCOPT 2025}{}{The modern theory of risk measures copes with uncertainty by considering multiple probability measures. While it is often assumed that a reference probability measure exists, under which all relevant probability measures are absolutely continuous, there are examples where this assumption does not hold, such as certain distributional robust functionals. In this talk, we introduce a novel class of dynamic risk measures that do not rely on this assumption. We will discuss its convexity, coherence, and time consistency properties.}
\talk{Duality results for Wasserstein Robust Optimization}{2025-05-17}{University of Wisconsin, Madison}{2025 AWM Research Symposium}{}{Distributionally robust optimization is a popular framework in data-driven decision making. We present a general duality result for Wasserstein distributionally robust optimization that holds for any Kantorovich transport cost, measurable loss function, and nominal probability distribution. Assuming an interchangeability principle inherent in existing duality results, our proof only uses one-dimensional convex analysis. In this talk, I will present the proof, and motivate the results using example with geometric interpretations. }
\talk{Quantitative Convergence Guarantees for the Mean-field Dispersion Process}{2025-05-13}{Denver}{SIAM Conference on Applications of Dynamical Systems (DS25)}{}{We study the discrete Fokker-Planck equation associated with the mean-field dynamics of a particle system called the dispersion process. For different regimes of the average number of particles per site, we establish various quantitative long-time convergence guarantees toward the global equilibrium, which is also confirmed by numerical simulations.}
\talk{Energy dissipation near the outflow boundary in the vanishing viscosity limit}{2025-04-15}{University of California, Los Angeles}{PDE and Analysis Seminar}{}{We consider the incompressible Navier-Stokes and Euler equations in a bounded domain with non-characteristic boundary condition, and study the energy dissipation near the outflow boundary in the zero-viscosity limit. We show that in a general setting, the energy dissipation rate is proportional to $\bar U \bar V ^2$, where $\bar U$ is the strength of the suction and $\bar V$ is the tangential component of the difference between Euler and Navier-Stokes on the outflow boundary. Moreover, we show that the enstrophy within a layer of order $\nu / \bar U$ is comparable with the total enstrophy. The rate of enstrophy production near the boundary is inversely proportional to $\nu$. This is based on joint work with Vincent Martinez, Anna Mazzucato, and Alexis Vasseur.}
\talk{Energy dissipation near the outflow boundary in the vanishing viscosity limit}{2025-04-09}{The University of Texas at Austin}{Analysis Seminar}{}{We consider the incompressible Navier-Stokes and Euler equations in a bounded domain with non-characteristic boundary condition, and study the energy dissipation near the outflow boundary in the zero-viscosity limit. We show that in a general setting, the energy dissipation rate is proportional to $\bar U \bar V ^2$, where $\bar U$ is the strength of the suction and $\bar V$ is the tangential component of the difference between Euler and Navier-Stokes on the outflow boundary. Moreover, we show that the enstrophy within a layer of order $\nu / \bar U$ is comparable with the total enstrophy. The rate of enstrophy production near the boundary is inversely proportional to $\nu$. This is based on joint work with Vincent Martinez, Anna Mazzucato, and Alexis Vasseur.}
\talk{Energy dissipation near the outflow boundary in the vanishing viscosity limit}{2025-03-27}{Princeton University}{Analysis of Fluids and Related Topics}{}{We consider the incompressible Navier-Stokes and Euler equations in a bounded domain with non-characteristic boundary condition, and study the energy dissipation near the outflow boundary in the zero-viscosity limit. We show that in a general setting, the energy dissipation rate is proportional to $\bar U \bar V ^2$, where $\bar U$ is the strength of the suction and $\bar V$ is the tangential component of the difference between Euler and Navier-Stokes on the outflow boundary. Moreover, we show that the enstrophy within a layer of order $\nu / \bar U$ is comparable with the total enstrophy. The rate of enstrophy production near the boundary is inversely proportional to $\nu$.}
\talk{New Estimates for Navier–Stokes and the Inviscid Limit Problem}{2024-11-26}{Institute for Advanced Study}{Analysis and Mathematical Physics}{}{In this talk, I will present several a priori interior and boundary trace estimates for the 3D incompressible Navier–Stokes equation, which recover and extend the current picture of higher derivative estimates in the mixed norm. Then we discuss the applications in the inviscid limit problem, with both characteristic and noncharacteristic boundary conditions. In particular, we provide estimates on layer separation and energy dissipation in the zero viscosity limit. This talk will be based on several work in collaboration with Alexis Vasseur, Vincent Martinez, and Anna Mazzucato.}
\talk{Energy dissipation near the outflow boundary in the vanishing viscosity limit}{2024-11-22}{CUNY Graduate Center}{Harmonic Analysis \& PDE Seminar}{}{We consider the incompressible Navier-Stokes and Euler equations in a bounded domain with non-characteristic boundary condition, and study the energy dissipation near the outflow boundary in the zero-viscosity limit. We show that in a general setting, the energy dissipation rate is proportional to $\bar U \bar V ^2$, where $\bar U$ is the strength of the suction and $\bar V$ is the strength of the shear. Moreover, we show that the rate of enstrophy production near the boundary is proportional to the Reynolds number.}
\talk{Beyond absolute continuity: a new class of dynamic risk measures}{2024-10-22}{Seattle}{Informs Annual Meeting 2024}{}{The modern theory of risk measures copes with uncertainty by considering multiple probability measures. While it is often assumed that a reference probability measure exists, under which all relevant probability measures are absolutely continuous, there are examples where this assumption does not hold, such as certain distributional robust functionals. In this talk, we introduce a novel class of dynamic risk measures that do not rely on this assumption. We will discuss its convexity, coherence, and time consistency properties.}
\talk{Energy dissipation near the outflow boundary in the vanishing viscosity limit}{2024-10-04}{Brown University}{PDE Seminar}{}{We consider the incompressible Navier-Stokes and Euler equations in a bounded domain with non-characteristic boundary condition, and study the energy dissipation near the outflow boundary in the zero-viscosity limit. We show that in a general setting, the energy dissipation rate is proportional to $\bar U \bar V ^2$, where $\bar U$ is the strength of the suction and $\bar V$ is the strength of the shear. Moreover, we show that the rate of enstrophy production near the boundary is proportional to the Reynolds number.}
\talk{Energy dissipation of Navier-Stokes equation with non-characteristic boundary condition}{2024-10-02}{Institute for Advanced Study}{Short Talks by Postdoctoral Members}{}{In this talk, I will discuss the inviscid limit problem. I will start with a general introduction of the Euler and Navier-Stokes system, and describe the problem of inviscid limit. I will cite the result of Kato (1984) which gives a necessary and sufficient condition for inviscid limit to hold. This will lead to a discussion on the zero-viscosity limit of energy dissipation. I will present a theorem of Vasseur—Yang (2023,2024) which gives an upper bound in the case of the impermeable and non-slip boundary condition, and an ongoing work which shows upper and lower bound in the case of the inflow-outflow condition.}
\talk{The role of risk measure in multistage distributionally robust stochastic optimization}{2024-08-22}{University of Illinois Urbana-Champaign}{Probabaility Seminar}{}{In this talk, we discuss the concept of risk measure, where the realms of optimization and finantial mathematics meets. New challenges arise in a dynamic environment and in data-driven settings. We will introduce a new class of dynamic risk measures with a good interpretability and time consistency. While it is often assumed that a reference probability measure exists, under which all relevant probability measures are absolutely continuous, there are examples where this assumption does not hold, such as certain distributional robust functionals. Our construction do not rely on this assumption.}
\talk{Layer separation and energy dissipation for 3D NSE at high Reynolds number}{2024-06-11}{Karlstad University}{EquaDiff 2024}{}{In this talk, we consider the 3D incompressible Navier-Stokes equation in a bounded domain, with a canonical example of Poiseuille flow in mind. We provide an unconditional upper bound for the boundary layer separation and energy dissipation of Leray–Hopf weak solutions, uniformly in high Reynolds numbers. We estimate layer separation by measuring the energy norm of the discrepancy between a (turbulent) low-viscosity Leray–Hopf solution and a fixed (laminar) regular Euler solution with similar initial conditions and body forces. This is accomplished by a new nonlinear boundary vorticity estimate, achieved along with several new trace estimates and higher derivative estimates using blow-up methods.}
\talk{A discrete Fokker-Planck equation for the dispersion process}{2024-05-14}{University of Minnesota}{Recent Advances in Nonlinear Partial Differential Equations}{}{We study the dispersion process on the complete graph introduced in the recent work under the mean-field framework. In contrast to the probabilistic approach, our focus is on the investigation of the large time behavior of solutions of the associated kinetic mean-field system of nonlinear ordinary differential equations (ODEs). We establish various analytical and quantitative convergence results for the long time behaviour of the mean-field system and related numerical illustrations are also provided.}
\talk{Layer separation and energy dissipation for 3D NSE at high Reynolds number}{2024-05-07}{New York University Abu Dhabi (zoom talk)}{SITE Research Center Talk Series}{}{In this talk, we consider the 3D incompressible Navier-Stokes equation in a bounded domain, with a canonical example of Poiseuille flow in mind. We provide an unconditional upper bound for the boundary layer separation and energy dissipation of Leray–Hopf weak solutions, uniformly in high Reynolds numbers. We estimate layer separation by measuring the energy norm of the discrepancy between a (turbulent) low-viscosity Leray–Hopf solution and a fixed (laminar) regular Euler solution with similar initial conditions and body force. This is accomplished by a new nonlinear boundary vorticity estimate.}
\talk{Vorticity interior trace estimates and higher derivative estimates via blow-up method}{2024-03-23}{Florida State University}{AMS Spring Southeastern Sectional Meeting}{}{We derive several nonlinear a priori trace estimates for the 3D incompressible Navier-Stokes equation, which extend the current picture of higher derivative estimates in the mixed norm. The main ingredient is the blow-up method and a novel averaging operator, which could apply to PDEs with scaling invariance and -regularity, possibly with a drift.}
\talk{Layer separation and energy dissipation for 3D NSE at high Reynolds number}{2024-02-19}{Johns Hopkins University (zoom talk)}{Simons Turbulence Seminar}{}{In this talk, we consider the 3D incompressible Navier-Stokes equation in a bounded domain, with a canonical example of Poiseuille flow in mind. We provide an unconditional upper bound for the boundary layer separation and energy dissipation of Leray–Hopf weak solutions, uniformly in high Reynolds numbers. We estimate layer separation by measuring the energy norm of the discrepancy between a (turbulent) low-viscosity Leray–Hopf solution and a fixed (laminar) regular Euler solution with similar initial conditions and body force. This is accomplished by a new nonlinear boundary vorticity estimate.}
\talk{Higher regularity and trace estimates for Navier-Stokes equation}{2023-11-13}{University of Chicago}{Calderón-Zygmund Analysis Seminar}{}{We derive several nonlinear a priori trace estimates for the 3D incompressible Navier-Stokes equation. They recover and extend the current picture of higher derivative estimates in the mixed norm. The main ingredient is the blow-up method and a novel averaging operator, which could apply to PDEs with scaling invariance and quantitative one-scale ε-regularity.}
\talk{Higher regularity and trace estimates for Navier-Stokes equation}{2023-11-02}{Purdue University}{PDE Seminar}{}{We derive several nonlinear a priori trace estimates for the 3D incompressible Navier-Stokes equation. They recover and extend the current picture of higher derivative estimates in the mixed norm. The main ingredient is the blow-up method and a novel averaging operator, which could apply to PDEs with scaling invariance and quantitative one-scale ε-regularity.}
\talk{Recent developments in the Navier-Stokes equation}{2023-10-12}{Johns Hopkins University}{Applied Mathematics and Statistics Seminar}{}{In this presentation, I will discuss recent progress regarding the regularity of the three-dimensional Navier-Stokes equation, a system of partial differential equations that models the behavior of fluids. While the full regularity of the 3D incompressible Navier-Stokes equation remains an outstanding open question, recently there have been significant breakthroughs in the fluid dynamics community. I will present new mathematical tools that provide deeper insights into the partial regularity of the Navier-Stokes equation and general supercritical systems. We derive nonlinear a priori estimates and trace estimates for the 3D incompressible Navier-Stokes equation, which extend the current picture of higher derivative estimates in the mixed norm. Additionally, I will demonstrate an intriguing application to the inviscid limit problem, which questions to what extent ideal fluids can model slightly viscous fluid.}
\talk{New a propri interior trace estimates on the 3D incompressible Navier-Stokes equation}{2023-10-06}{University of Nebraska-Lincoln}{8\textsuperscript{th} Annual Meeting of the SIAM Central States Section}{}{We derive several nonlinear a priori trace estimates for the 3D incompressible Navier-Stokes equation, which extend the current picture of higher derivative estimates in the mixed norm. The main ingredient is the blow-up method and a novel averaging operator, which could apply to PDEs with scaling invariance and ε-regularity, possibly with a drift.}
\talk{Vorticity estimates for the 3D incompressible Navier-Stokes equation}{2023-08-21}{Waseda University (zoom talk)}{10\textsuperscript{th} International Congress on Industrial and Applied Mathematics}{}{We show some a priori regularity estimates for the vorticity and its trace in the three-dimensional incompressible Navier-Stokes equation. These a priori estimates are obtained via the blow-up method and a novel averaging operator. The averaging operator can be used to provide regularity and trace estimates for PDEs with ε-regularity.}
\talk{Trace estimates of 3D NSE via blow-up}{2023-07-07}{Chinese Academy of Sciences}{PDE and Applications Seminar}{}{We derive several nonlinear a priori trace estimates for the 3D incompressible Navier-Stokes equation, which extend the current picture of higher derivative estimates in the mixed norm. The main ingredient is the blow-up method and a novel averaging operator, which could apply to PDEs with scaling invariance and ε-regularity, possibly with a drift.}
\talk{Layer separation, anomalous dissipation, and drag force in the inviscid limit for 3D NSE}{2023-05-06}{University of Notre Dame}{Midwest PDE Seminar}{}{We provide an unconditional upper bound for the boundary layer separation, anomalous dissi- pation, and the work of drag force in the zero viscosity limit, of Leray–Hopf weak solutions to the 3D incompressible Navier-Stokes equation in a smooth bounded domain. Layer separation refers to the discrepancy between a (turbulent) low-viscosity Leray–Hopf solution and a fixed (laminar) regular Euler solution with similar initial conditions and body force. In addition, we show the boundedness of the drag coefficient with a high Reynolds number.}
\talk{Layer separation for the 3D Navier-Stokes equation in a bounded domain}{2023-01-09}{University of Chicago}{Calderón-Zygmund Analysis Seminar}{}{We provide an unconditional $L^2$ upper bound for the boundary layer separation of 3D Leray-Hopf solutions in a smooth bounded domain. By layer separation, we mean the discrepency between a (turbulent) low-viscosity Leray-Hopf solution $u^\nu$ and a fixed (laminar) regular Euler solution $\bar u$ with initial conditions close in $L^2$. Layer separation appears in physical and numerical experiments near the boundary, and we bound it asymptotically by $C \lVert\bar u\rVert_{L^\infty}^3 t$. This extends the previous result when the Euler solution is a regular shear in a finite channel. The key estimate is to control the boundary vorticity in a way that does not degenerate in the vanishing viscosity limit. This is joint work with Alexis Vasseur.}
\talk{Layer separation for the 3D Navier-Stokes equation in a bounded domain}{2022-12-16}{Chinese Academy of Sciences (zoom talk)}{PDE and Applications Seminar}{}{We provide an unconditional $L^2$ upper bound for the boundary layer separation of 3D Leray--Hopf solutions in a smooth bounded domain. By layer separation, we mean the discrepancy between a (turbulent) low-viscosity Leray--Hopf solution $u^\nu$ and a fixed (laminar) regular Euler solution $\bar u$ with initial conditions close in $L^2$. Layer separation appears in physical and numerical experiments near the boundary, and we bound it asymptotically by $C \|\bar u\|_{L^\infty}^3 t$. This extends the previous result when the Euler solution is a regular shear in a finite channel. The key estimate is to control the boundary vorticity in a way that does not degenerate in the vanishing viscosity limit.}
\talk{Layer separation for the 3D Navier-Stokes equation in a bounded domain}{2022-11-03}{Princeton University}{Analysis of Fluids and Related Topics}{}{We provide an unconditional $L^2$ upper bound for the boundary layer separation of 3D Leray-Hopf solutions in a smooth bounded domain. By layer separation, we mean the discrepency between a (turbulent) low-viscosity Leray-Hopf solution $u^\nu$ and a fixed (laminar) regular Euler solution $\bar u$ with initial conditions close in $L^2$. Layer separation appears in physical and numerical experiments near the boundary, and we bound it asymptotically by $C \lVert\bar u\rVert_{L^\infty}^3 t$. This extends the previous result when the Euler solution is a regular shear in a finite channel. The key estimate is to control the boundary vorticity in a way that does not degenerate in the vanishing viscosity limit. This is joint work with Alexis Vasseur.}
\talk{Distributionally Robust End-to-end Learning with Side Information}{2022-10-16}{Indianapolis}{Informs Annual Meeting 2022}{}{We consider data-driven decision-making under uncertainty with side information, in which exogenous covariates data are available to reveal partial information about random problem parameters and thus facilitate better decisions. The decision maker's goal is to find an optimal end-to-end policy that directly outputs a decision for any new context. We propose a distributionally robust formulation based on the causal transport distance, which preserves the conditional information structure of random problem parameters given the value of covariates. We derive a dual reformulation and study its regularization effect. We prove that the infinite-dimensional policy optimization admits a finite-dimensional convex programming equivalent reformulation. This result renders a new class of optimal policy for adjustable robust optimization.}
\talk{New Estimates on the Second Derivatives of the 3D Navier-Stokes Equation}{2021-05-02}{San Francisco State University (zoom talk)}{AMS Spring Western Sectional Meeting}{}{In this talk we present $L ^{4/3,q}$ local integrability of the second spatial derivatives of suitable solutions to the Navier-Stokes equation for any $q > \frac43$. This joint work with A. Vasseur improves the current result $L ^{4/3, \infty}$ (Lions, 1996), and it is based on a blow-up technique using the universal scaling along approximated Lagrangian trajectories. Locally, we can obtain any regularity of the vorticity without any a priori knowledge of the pressure. The local-to-global step uses a recently constructed maximal function for transport equations.}
\talk{New Estimates on the Second Derivatives of the 3D Navier-Stokes Equation}{2021-03-20}{Brown University (zoom talk)}{AMS Spring Eastern Sectional Meeting}{}{In this talk we present $L ^{4/3,q}$ local integrability of the second spatial derivatives of suitable solutions to the Navier-Stokes equation for any $q > \frac43$. This joint work with A. Vasseur improves the current result $L ^{4/3, \infty}$ (Lions, 1996), and it is based on a blow-up technique using the universal scaling along approximated Lagrangian trajectories. Locally, we can obtain any regularity of the vorticity without any a priori knowledge of the pressure. The local-to-global step uses a recently constructed maximal function for transport equations.}


% \subsection{Seminar Talks}
% \renewcommand{\talk}[5]{\cvitem{\DTMdate{#2}}{#4, \textit{#1}}}
% \talk{}{2024-06-10}{Karlstad University}{EquaDiff 2024}{}
\talk{Partial regularity results for the three-dimensional incompressible Navier-Stokes equation}{2022-03-25}{The University of Texas at Austin (hybrid talk)}{Ph.D. Defense Talk}{We show a series of works of some regularity results on the incompressible Navier--Stokes equation in dimension three. Using the blow-up method, we estimate the higher regularity in the Lorentz norm for smooth solutions to the Navier--Stokes equation. In particular, we show a second derivative estimate for suitable weak solutions, which improves the currently known regularity. We construct a maximal function associated with geometric objects that we call skewed cylinders, appearing in inviscid flows like the Eulerian cylinders around the Lagrangian trajectories. We also apply the blow-up method to estimate the boundary vorticity, which enables us to achieve an unconditional control of the layer separation of Leray--Hopf solutions from a steady shear flow in a finite periodic channel. }
\talk{Inviscid Limit and Boundary Layer Separation}{2021-10-15}{The University of Texas at Austin (zoom talk)}{Junior Analysis Seminar}{We discuss the limiting behavior of a Leray-Hopf solution to the Navier-Stokes equation in dimension 2 and 3 as viscosity vanishes and initial velocity profile converges to a constant shear flow. Without further assumption, it is a major open question whether a solution to the Navier-Stokes equation converges to a solution to the Euler equation. In this talk, we will present a new estimate on the discrepency between a Navier-Stokes solution and a static Euler solution albeit the inviscid limit may fail. This is the first unconditional result from the positive side, and the bound is near sharp as suggested by convex integration.}
\talk{Shapley Extension and Application}{2021-04-09}{The University of Texas at Austin (zoom talk)}{Junior Analysis Seminar}{In this talk, I will introduce a recently constructed extension operator, called the "Shapley extension". This operator gives a Lipschitz extension for a function defined on a discrete subset of metric space to the whole space. The extension is based on the Shapley theorem, which comes from the Nash equilibrium in the game theory. We will give an application of the Shapley extension in 1-Wasserstein robust optimization problem. This is a joint work with Gao and Zhang.}
\talk{Distributionally Robust Optimization}{2020-10-09}{The University of Texas at Austin (zoom talk)}{Junior Analysis Seminar}{We want to estimate the expectation of a random variable when the underlying probability measure is unknown. The true probability can differ from some nominal distribution by a distance R, where the distance between distributions can be characterized by the optimal mass transport, for instance Wasserstein distance. In this talk, we will explore the duality principle for this estimate, and we can discuss the worst case distribution for the discrete case if time permits.}
\talk{Oscillatory Integral and Decay of Dispersive Equation}{2020-04-10}{The University of Texas at Austin (zoom talk)}{Junior Analysis Seminar}{In this talk, we will introduce the oscillatory integral of the first kind $\int \exp(i t \phi (x)) \psi (x) dx$. We present the Van der Corput lemma, and the method of stationary phase. As an application in PDE, we will analyze the decay rate for a family of 1-dimensional linear dispersive equations. If time permits, we can look at a stability result for the inviscid Boussinesq system.}
\talk{A New Covering Lemma and its Application in 3D Incompressible Navier-Stokes Equations}{2020-03-27}{The University of Texas at Austin (zoom talk)}{Ph.D. Candidacy Talk}{We want to give second derivative estimate for the suitable weak solutions to three dimensional Navier-Stokes equations with $L ^2$ initial data. Right now it is known that $\nabla ^2 u$ is in the space of $L ^{\frac43, \infty}$ homogeneous in space-time. The limiting space $L ^{\frac43}$, however, is unknown. In 2014, Choi and Vasseur gave an alternating proof for $L ^\frac43$ weak integrability, where they used blow-up techniques along trajectories and De Giorgi methods. To do local analysis, they investigate solutions near a Eulerian cylindrical neighborhood of a given point. In this talk, I present a covering lemma for these Eulerian cylinders, which could bridge the gap between $L ^\frac43$ weak and $L ^\frac43$. We are now working on a refinement of the local analysis, which will be able to give the strict $L ^\frac43$ integrability locally. }
\talk{Introduction to Hardy Space}{2019-11-11}{The University of Texas at Austin}{Harmonic Analysis Reading Seminar}{I will present several equivalent definition of Hardy space. Then I will try to prove their equivalence as much as I can. Reference: Grafakos, Section 6.4.1, 6.4.2; Stein, Section 3.1.}
\talk{A Covering Lemma for Fluid and Application to Navier-Stokes Equation}{2019-11-01}{The University of Texas at Austin}{Junior Analysis Seminar}{For a fluid equation, one way to obtain higher derivatives estimate is by scaling. In particular, we smoothen the flow, zoom in and center at the regularized trajectories, and obtain interior regularity in local frame, say in a unit cube. These cubes may look very different from each other in the global frame. I will present an interesting Vitali-type covering lemma for these cubes, and as a consequence I will show $W^{2,p}$ boundedness of suitable solutions to Navier-Stokes equation for $p < 4/3$.}
\talk{Manifold-valued Semimartingales and their Quadratic Variation}{2019-10-11}{The University of Texas at Austin}{Stochastic Calculus in Manifold Reading Seminar}{I will go through the Chapter III of Stochastic Calculus in Manifold by Michel Emery. I will introduce the definition of semimartingales in manifold, and define quadratic variation for semimartingale.}
\talk{Regularity for a Local-Nonlocal Transmission Problem}{2019-04-12}{The University of Texas at Austin}{Junior Analysis Seminar}{I would like to share the thesis of Dennis Kriventsov when he graduated from here four years ago. In his paper, he proposed a model for an SQG equation with different dissipation operators on two domains, corresponding to the ocean and the land, that partition the whole space. He constructed admissible weak solutions, and used De Giorgi method to show their Hölder continuity. He also bootstrapped to the optimal regularity in some cases. In this talk, I will focus on his De Giorgi part that proves the initial $C^α$ regularity, and show how he used a reflection method to deal with the main difficulty, the discrepancy in scaling.}
\talk{An Introduction to Stochastic Integrals}{2018-04-20}{The University of Texas at Austin}{Sophex Seminar}{As the only one taking the probability prelim courses in our cohort, I want to share with everyone interesting things that are happening in the world of randomness. This talk will start with the intuition of stochastic integral, then introduce very informally the idea of martingale, and finally move to the core of SDE (stochastic differential equation): the famous Itô's lemma.}
\talk{Stability and instability of shear flow in a rotating system}{2017-09-22}{The University of Texas at Austin}{Sophex Seminar}{When studying the large-scale movement of the ocean water and atmosphere, earth rotation may affect the stability of the fluid significantly. Here we discuss the linear stability of an homogeneous inviscid incompressible shear flow on a rotating frame near equator. Coriolis effect will be taken into consideration. The goal of this seminar is to illustrate how to show linear stability and instability of such a flow. As an example, we present a thorough analysis of the sinusoidal shear flow, and point out a mistake in Kuo’s (1973) book. This is a joint work with Dr. Hao Zhu and our advisor Prof. Zhiwu Lin at Georgia Tech.}
\talk{Inviscid Damping of Couette Flow in a Stratified Fluid}{2017-04-26}{Georgia Institute of Technology}{Fluid Mechanics Seminar}{We study the inviscid damping of Couette flow with an exponentially stratified density. The optimal decay rates of the velocity field and the density are obtained for general perturbations with minimal regularity. For Boussinesq approximation model, the decay rates we get are consistent with the previous results in the literature. We also study the decay rates for the full Euler equations of stratified fluids, which were not studied before. For both models, the decay rates depend on the Richardson number in a very similar way. Besides, we also study the dispersive decay due to the exponential stratification when there is no shear. }


\newpage
\section{Service}

\mysubsection{Seminar Organizing}
\cventry{2023--2024}{Calder\'on--Zygmund Analysis Seminar}{University of Chicago}{}{}{}
\cventry{2019}{Harmonic Analysis Reading Seminar}{University of Texas at Austin}{}{}{}

\mysubsection{Journal Refereeing}
\cvitem{}{SIAM Journal on Mathematical Analysis}
\cvitem{}{Journal of Differential Equations}
\cvitem{}{Mathematical Programming}
\cvitem{}{Nonlinearity}

\section{Teaching Experience}

\newcommand{\teach}[4]{\cvitem{#3 #4}{#1 -- #2}}
\mysubsection{Instructor}
\teach{Math 18500}{Mathematical Methods in the Physical Sciences III}{Spring}{2024}
\teach{Math 27300}{Basic Theory of Ordinary Differential Equations}{Winter}{2024}
\teach{Math 18500}{Mathematical Methods in the Physical Sciences III}{Spring}{2023}
\teach{Math 20400}{Analysis in $\mathbb R ^n$ II}{Winter}{2023}
\teach{Math 18500}{Mathematical Methods in the Physical Sciences III}{Fall}{2022}


\mysubsection{Teaching Assistant}
\teach{M383D}{Methods of Applied Mathematics II}{Spring}{2021}
\teach{M427J}{Differential Equations with Linear Algebra}{}{2019-2020}
% \teach{M427J}{Differential Equations with Linear Algebra}{Spring}{2020}
% \teach{M427J}{Differential Equations with Linear Algebra}{Fall}{2019}
% \teach{M427J}{Differential Equations with Linear Algebra}{Spring}{2019}
\teach{M427L}{Advanced Calculus for Applications II}{Fall}{2018}
\teach{M408D}{Sequences, Series, and Multivariable Calculus}{Spring}{2018}
\teach{M408K}{Differential Calculus}{Fall}{2017}


\section{Mentoring Undergraduate Students}

\newcommand{\drp}[4]{\cvitem{#3 #4}{#1, \textit{#2}}}

\mysubsection{Research Experience for Undergraduates}
\drp{Minh Pham}{on the topic of Nonlinear Integral Equations}{Summer}{2024}
\drp{Michael Lee}{on the topic of Partial Regularity Theory of Navier-Stokes Equations}{Summer}{2024}
\drp{Zichen Lu}{on the topic of Inviscid Damping of Euler Equations}{Summer}{2024}


\mysubsection{Reading Course}
\drp{Dante Strollo and Jakob Wellington}{on the topic of Fourier Analysis}{Spring}{2024}


\mysubsection{Directed Reading Program}
\drp{Ariana Qin}{on the topic of Stochastic Calculus and Stochastic Control}{Summer}{2024}
\drp{Ariana Qin}{on the topic of Optimal Control Theory}{Spring}{2024}
\drp{Kyle Alkire}{on the topic of Schauder Theory in Elliptic Equations}{Fall}{2021}
\drp{Yongqi Pang}{on the topic of Statistics and Data Analysis}{Spring}{2020}
\drp{Trey Minor}{on the topic of Differential Equations and Dynamical Systems}{Spring}{2019}
\drp{Yan Cheng}{on the topic of Probability and Martingales}{Spring}{2018}



% 
%----------------------------------------------------------------------------------------
%	COMPUTER SKILLS SECTION
%----------------------------------------------------------------------------------------

\section{Computer skills}

\cvitem{Advanced}{Java, \LaTeX, Matlab, Mathematica, Processing, Python, RAPTOR}
\cvitem{Intermediate}{C++, C\#, HTML, JavaScript, Swift}
\cvitem{Basic}{CSS, Objective C, R, Ruby}

%----------------------------------------------------------------------------------------
%	COMMUNICATION SKILLS SECTION
%----------------------------------------------------------------------------------------

% \section{Communication Skills}

% \cvitem{2010}{Oral Presentation at the California Business Conference}
% \cvitem{2009}{Poster at the Annual Business Conference in Oregon}

%----------------------------------------------------------------------------------------
%	LANGUAGES SECTION
%----------------------------------------------------------------------------------------

\section{Languages}

\cvitemwithcomment{Fluent}{Chinese (mothertongue), English}{}
\cvitemwithcomment{Intermediate}{French (academic reading)}{}
\cvitemwithcomment{Basic}{Japanese, Spanish}{}

%----------------------------------------------------------------------------------------
%	INTERESTS SECTION
%----------------------------------------------------------------------------------------

% \section{Interests}

% \renewcommand{\listitemsymbol}{-~} % Changes the symbol used for lists

% \cvlistdoubleitem{Piano}{Chess}
% \cvlistdoubleitem{Cooking}{Dancing}
% \cvlistitem{Running}

%----------------------------------------------------------------------------------------

\end{document}


%% end of file `template.tex'.
