\talk{}{2024-06-10}{Karlstad University}{EquaDiff 2024}{}
\talk{Vorticity interior trace estimates and higher derivative estimates via blow-up method}{2024-03-23}{Florida State University}{AMS Spring Southeastern Sectional Meeting}{We derive several nonlinear a priori trace estimates for the 3D incompressible Navier-Stokes equation, which extend the current picture of higher derivative estimates in the mixed norm. The main ingredient is the blow-up method and a novel averaging operator, which could apply to PDEs with scaling invariance and -regularity, possibly with a drift.}
\talk{Layer separation and energy dissipation for 3D NSE at high Reynolds number}{2024-02-19}{Johns Hopkins University (zoom talk)}{Simons Turbulence Seminar}{In this talk, we consider the 3D incompressible Navier-Stokes equation in a bounded domain, with a canonical example of Poiseuille flow in mind. We provide an unconditional upper bound for the boundary layer separation and energy dissipation of Leray–Hopf weak solutions, uniformly in high Reynolds numbers. We estimate layer separation by measuring the energy norm of the discrepancy between a (turbulent) low-viscosity Leray–Hopf solution and a fixed (laminar) regular Euler solution with similar initial conditions and body force. This is accomplished by a new nonlinear boundary vorticity estimate.}
\talk{Higher regularity and trace estimates for Navier-Stokes equation}{2023-11-13}{University of Chicago}{Calderón-Zygmund Analysis Seminar}{We derive several nonlinear a priori trace estimates for the 3D incompressible Navier-Stokes equation. They recover and extend the current picture of higher derivative estimates in the mixed norm. The main ingredient is the blow-up method and a novel averaging operator, which could apply to PDEs with scaling invariance and quantitative one-scale ε-regularity.}
\talk{Higher regularity and trace estimates for Navier-Stokes equation}{2023-11-02}{Purdue University}{PDE Seminar}{We derive several nonlinear a priori trace estimates for the 3D incompressible Navier-Stokes equation. They recover and extend the current picture of higher derivative estimates in the mixed norm. The main ingredient is the blow-up method and a novel averaging operator, which could apply to PDEs with scaling invariance and quantitative one-scale ε-regularity.}
\talk{Recent developments in the Navier-Stokes equation}{2023-10-12}{Johns Hopkins University}{Applied Mathematics and Statistics Seminar}{In this presentation, I will discuss recent progress regarding the regularity of the three-dimensional Navier-Stokes equation, a system of partial differential equations that models the behavior of fluids. While the full regularity of the 3D incompressible Navier-Stokes equation remains an outstanding open question, recently there have been significant breakthroughs in the fluid dynamics community. I will present new mathematical tools that provide deeper insights into the partial regularity of the Navier-Stokes equation and general supercritical systems. We derive nonlinear a priori estimates and trace estimates for the 3D incompressible Navier-Stokes equation, which extend the current picture of higher derivative estimates in the mixed norm. Additionally, I will demonstrate an intriguing application to the inviscid limit problem, which questions to what extent ideal fluids can model slightly viscous fluid.}
\talk{New a propri interior trace estimates on the 3D incompressible Navier-Stokes equation}{2023-10-06}{University of Nebraska-Lincoln}{8\textsuperscript{th} Annual Meeting of the SIAM Central States Section}{We derive several nonlinear a priori trace estimates for the 3D incompressible Navier-Stokes equation, which extend the current picture of higher derivative estimates in the mixed norm. The main ingredient is the blow-up method and a novel averaging operator, which could apply to PDEs with scaling invariance and ε-regularity, possibly with a drift.}
\talk{Vorticity estimates for the 3D incompressible Navier-Stokes equation}{2023-08-21}{Waseda University (zoom talk)}{10\textsuperscript{th} International Congress on Industrial and Applied Mathematics}{We show some a priori regularity estimates for the vorticity and its trace in the three-dimensional incompressible Navier-Stokes equation. These a priori estimates are obtained via the blow-up method and a novel averaging operator. The averaging operator can be used to provide regularity and trace estimates for PDEs with ε-regularity.}
\talk{Trace estimates of 3D NSE via blow-up}{2023-07-07}{Chinese Academy of Sciences}{PDE and Applications Seminar}{We derive several nonlinear a priori trace estimates for the 3D incompressible Navier-Stokes equation, which extend the current picture of higher derivative estimates in the mixed norm. The main ingredient is the blow-up method and a novel averaging operator, which could apply to PDEs with scaling invariance and ε-regularity, possibly with a drift.}
\talk{Layer separation, anomalous dissipation, and drag force in the inviscid limit for 3D NSE}{2023-05-06}{University of Notre Dame}{Midwest PDE Seminar}{We provide an unconditional upper bound for the boundary layer separation, anomalous dissi- pation, and the work of drag force in the zero viscosity limit, of Leray–Hopf weak solutions to the 3D incompressible Navier-Stokes equation in a smooth bounded domain. Layer separation refers to the discrepancy between a (turbulent) low-viscosity Leray–Hopf solution and a fixed (laminar) regular Euler solution with similar initial conditions and body force. In addition, we show the boundedness of the drag coefficient with a high Reynolds number.}
\talk{Layer separation for the 3D Navier-Stokes equation in a bounded domain}{2023-01-09}{University of Chicago}{Calderón-Zygmund Analysis Seminar}{We provide an unconditional $L^2$ upper bound for the boundary layer separation of 3D Leray-Hopf solutions in a smooth bounded domain. By layer separation, we mean the discrepency between a (turbulent) low-viscosity Leray-Hopf solution $u^\nu$ and a fixed (laminar) regular Euler solution $\bar u$ with initial conditions close in $L^2$. Layer separation appears in physical and numerical experiments near the boundary, and we bound it asymptotically by $C \lVert\bar u\rVert_{L^\infty}^3 t$. This extends the previous result when the Euler solution is a regular shear in a finite channel. The key estimate is to control the boundary vorticity in a way that does not degenerate in the vanishing viscosity limit. This is joint work with Alexis Vasseur.}
\talk{Layer separation for the 3D Navier-Stokes equation in a bounded domain}{2022-12-16}{Chinese Academy of Sciences (zoom talk)}{PDE and Applications Seminar}{We provide an unconditional $L^2$ upper bound for the boundary layer separation of 3D Leray--Hopf solutions in a smooth bounded domain. By layer separation, we mean the discrepancy between a (turbulent) low-viscosity Leray--Hopf solution $u^
u$ and a fixed (laminar) regular Euler solution $ar u$ with initial conditions close in $L^2$. Layer separation appears in physical and numerical experiments near the boundary, and we bound it asymptotically by $C \|ar u\|_{L^\infty}^3 t$. This extends the previous result when the Euler solution is a regular shear in a finite channel. The key estimate is to control the boundary vorticity in a way that does not degenerate in the vanishing viscosity limit.}
\talk{Layer separation for the 3D Navier-Stokes equation in a bounded domain}{2022-11-03}{Princeton University}{Analysis of Fluids and Related Topics}{We provide an unconditional $L^2$ upper bound for the boundary layer separation of 3D Leray-Hopf solutions in a smooth bounded domain. By layer separation, we mean the discrepency between a (turbulent) low-viscosity Leray-Hopf solution $u^\nu$ and a fixed (laminar) regular Euler solution $\bar u$ with initial conditions close in $L^2$. Layer separation appears in physical and numerical experiments near the boundary, and we bound it asymptotically by $C \lVert\bar u\rVert_{L^\infty}^3 t$. This extends the previous result when the Euler solution is a regular shear in a finite channel. The key estimate is to control the boundary vorticity in a way that does not degenerate in the vanishing viscosity limit. This is joint work with Alexis Vasseur.}
\talk{Distributionally Robust End-to-end Learning with Side Information}{2022-10-16}{Indianapolis}{Informs Annual Meeting 2022}{We consider data-driven decision-making under uncertainty with side information, in which exogenous covariates data are available to reveal partial information about random problem parameters and thus facilitate better decisions. The decision maker's goal is to find an optimal end-to-end policy that directly outputs a decision for any new context. We propose a distributionally robust formulation based on the causal transport distance, which preserves the conditional information structure of random problem parameters given the value of covariates. We derive a dual reformulation and study its regularization effect. We prove that the infinite-dimensional policy optimization admits a finite-dimensional convex programming equivalent reformulation. This result renders a new class of optimal policy for adjustable robust optimization.}
\talk{Partial regularity results for the three-dimensional incompressible Navier-Stokes equation}{2022-03-25}{The University of Texas at Austin (hybrid talk)}{Ph.D. Defense Talk}{We show a series of works of some regularity results on the incompressible Navier--Stokes equation in dimension three. Using the blow-up method, we estimate the higher regularity in the Lorentz norm for smooth solutions to the Navier--Stokes equation. In particular, we show a second derivative estimate for suitable weak solutions, which improves the currently known regularity. We construct a maximal function associated with geometric objects that we call skewed cylinders, appearing in inviscid flows like the Eulerian cylinders around the Lagrangian trajectories. We also apply the blow-up method to estimate the boundary vorticity, which enables us to achieve an unconditional control of the layer separation of Leray--Hopf solutions from a steady shear flow in a finite periodic channel. }
\talk{Inviscid Limit and Boundary Layer Separation}{2021-10-15}{The University of Texas at Austin (zoom talk)}{Junior Analysis Seminar}{We discuss the limiting behavior of a Leray-Hopf solution to the Navier-Stokes equation in dimension 2 and 3 as viscosity vanishes and initial velocity profile converges to a constant shear flow. Without further assumption, it is a major open question whether a solution to the Navier-Stokes equation converges to a solution to the Euler equation. In this talk, we will present a new estimate on the discrepency between a Navier-Stokes solution and a static Euler solution albeit the inviscid limit may fail. This is the first unconditional result from the positive side, and the bound is near sharp as suggested by convex integration.}
\talk{New Estimates on the Second Derivatives of the 3D Navier-Stokes Equation}{2021-05-02}{San Francisco State University (zoom talk)}{AMS Spring Western Sectional Meeting}{In this talk we present $L ^{4/3,q}$ local integrability of the second spatial derivatives of suitable solutions to the Navier-Stokes equation for any $q > \frac43$. This joint work with A. Vasseur improves the current result $L ^{4/3, \infty}$ (Lions, 1996), and it is based on a blow-up technique using the universal scaling along approximated Lagrangian trajectories. Locally, we can obtain any regularity of the vorticity without any a priori knowledge of the pressure. The local-to-global step uses a recently constructed maximal function for transport equations.}
\talk{Shapley Extension and Application}{2021-04-09}{The University of Texas at Austin (zoom talk)}{Junior Analysis Seminar}{In this talk, I will introduce a recently constructed extension operator, called the "Shapley extension". This operator gives a Lipschitz extension for a function defined on a discrete subset of metric space to the whole space. The extension is based on the Shapley theorem, which comes from the Nash equilibrium in the game theory. We will give an application of the Shapley extension in 1-Wasserstein robust optimization problem. This is a joint work with Gao and Zhang.}
\talk{New Estimates on the Second Derivatives of the 3D Navier-Stokes Equation}{2021-03-20}{Brown University (zoom talk)}{AMS Spring Eastern Sectional Meeting}{In this talk we present $L ^{4/3,q}$ local integrability of the second spatial derivatives of suitable solutions to the Navier-Stokes equation for any $q > \frac43$. This joint work with A. Vasseur improves the current result $L ^{4/3, \infty}$ (Lions, 1996), and it is based on a blow-up technique using the universal scaling along approximated Lagrangian trajectories. Locally, we can obtain any regularity of the vorticity without any a priori knowledge of the pressure. The local-to-global step uses a recently constructed maximal function for transport equations.}
\talk{Distributionally Robust Optimization}{2020-10-09}{The University of Texas at Austin (zoom talk)}{Junior Analysis Seminar}{We want to estimate the expectation of a random variable when the underlying probability measure is unknown. The true probability can differ from some nominal distribution by a distance R, where the distance between distributions can be characterized by the optimal mass transport, for instance Wasserstein distance. In this talk, we will explore the duality principle for this estimate, and we can discuss the worst case distribution for the discrete case if time permits.}
\talk{Oscillatory Integral and Decay of Dispersive Equation}{2020-04-10}{The University of Texas at Austin (zoom talk)}{Junior Analysis Seminar}{In this talk, we will introduce the oscillatory integral of the first kind $\int \exp(i t \phi (x)) \psi (x) dx$. We present the Van der Corput lemma, and the method of stationary phase. As an application in PDE, we will analyze the decay rate for a family of 1-dimensional linear dispersive equations. If time permits, we can look at a stability result for the inviscid Boussinesq system.}
\talk{A New Covering Lemma and its Application in 3D Incompressible Navier-Stokes Equations}{2020-03-27}{The University of Texas at Austin (zoom talk)}{Ph.D. Candidacy Talk}{We want to give second derivative estimate for the suitable weak solutions to three dimensional Navier-Stokes equations with $L ^2$ initial data. Right now it is known that $\nabla ^2 u$ is in the space of $L ^{\frac43, \infty}$ homogeneous in space-time. The limiting space $L ^{\frac43}$, however, is unknown. In 2014, Choi and Vasseur gave an alternating proof for $L ^\frac43$ weak integrability, where they used blow-up techniques along trajectories and De Giorgi methods. To do local analysis, they investigate solutions near a Eulerian cylindrical neighborhood of a given point. In this talk, I present a covering lemma for these Eulerian cylinders, which could bridge the gap between $L ^\frac43$ weak and $L ^\frac43$. We are now working on a refinement of the local analysis, which will be able to give the strict $L ^\frac43$ integrability locally. }
\talk{Introduction to Hardy Space}{2019-11-11}{The University of Texas at Austin}{Harmonic Analysis Reading Seminar}{I will present several equivalent definition of Hardy space. Then I will try to prove their equivalence as much as I can. Reference: Grafakos, Section 6.4.1, 6.4.2; Stein, Section 3.1.}
\talk{A Covering Lemma for Fluid and Application to Navier-Stokes Equation}{2019-11-01}{The University of Texas at Austin}{Junior Analysis Seminar}{For a fluid equation, one way to obtain higher derivatives estimate is by scaling. In particular, we smoothen the flow, zoom in and center at the regularized trajectories, and obtain interior regularity in local frame, say in a unit cube. These cubes may look very different from each other in the global frame. I will present an interesting Vitali-type covering lemma for these cubes, and as a consequence I will show $W^{2,p}$ boundedness of suitable solutions to Navier-Stokes equation for $p < 4/3$.}
\talk{Manifold-valued Semimartingales and their Quadratic Variation}{2019-10-11}{The University of Texas at Austin}{Stochastic Calculus in Manifold Reading Seminar}{I will go through the Chapter III of Stochastic Calculus in Manifold by Michel Emery. I will introduce the definition of semimartingales in manifold, and define quadratic variation for semimartingale.}
\talk{Regularity for a Local-Nonlocal Transmission Problem}{2019-04-12}{The University of Texas at Austin}{Junior Analysis Seminar}{I would like to share the thesis of Dennis Kriventsov when he graduated from here four years ago. In his paper, he proposed a model for an SQG equation with different dissipation operators on two domains, corresponding to the ocean and the land, that partition the whole space. He constructed admissible weak solutions, and used De Giorgi method to show their Hölder continuity. He also bootstrapped to the optimal regularity in some cases. In this talk, I will focus on his De Giorgi part that proves the initial $C^α$ regularity, and show how he used a reflection method to deal with the main difficulty, the discrepancy in scaling.}
\talk{An Introduction to Stochastic Integrals}{2018-04-20}{The University of Texas at Austin}{Sophex Seminar}{As the only one taking the probability prelim courses in our cohort, I want to share with everyone interesting things that are happening in the world of randomness. This talk will start with the intuition of stochastic integral, then introduce very informally the idea of martingale, and finally move to the core of SDE (stochastic differential equation): the famous Itô's lemma.}
\talk{Stability and instability of shear flow in a rotating system}{2017-09-22}{The University of Texas at Austin}{Sophex Seminar}{When studying the large-scale movement of the ocean water and atmosphere, earth rotation may affect the stability of the fluid significantly. Here we discuss the linear stability of an homogeneous inviscid incompressible shear flow on a rotating frame near equator. Coriolis effect will be taken into consideration. The goal of this seminar is to illustrate how to show linear stability and instability of such a flow. As an example, we present a thorough analysis of the sinusoidal shear flow, and point out a mistake in Kuo’s (1973) book. This is a joint work with Dr. Hao Zhu and our advisor Prof. Zhiwu Lin at Georgia Tech.}
\talk{Inviscid Damping of Couette Flow in a Stratified Fluid}{2017-04-26}{Georgia Institute of Technology}{Fluid Mechanics Seminar}{We study the inviscid damping of Couette flow with an exponentially stratified density. The optimal decay rates of the velocity field and the density are obtained for general perturbations with minimal regularity. For Boussinesq approximation model, the decay rates we get are consistent with the previous results in the literature. We also study the decay rates for the full Euler equations of stratified fluids, which were not studied before. For both models, the decay rates depend on the Richardson number in a very similar way. Besides, we also study the dispersive decay due to the exponential stratification when there is no shear. }
